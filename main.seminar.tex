\documentclass[a4paper,11pt,bibliography=totoc,listof=totoc,headinclude=true,cleardoublepage=empty,oneside]{NumPDEsThesis}
% Option "oneside" für einseitigen Druck. Weglassen, falls die Arbeit doppelseitig gedruckt wird

% Lade Literaturdatenbank
\addbibresource{literature.bib}

% Lade weitere Pakete
\usepackage{NumPDEsMacros} % in NumPDEsMacros.sty sind einige nützliche Befehle definiert

% Bevor die Arbeit gedruckt wird, sollten die Farben ausgeschaltet werden, indem die folgende Zeile auskommentiert wird.
% \hypersetup{colorlinks=true,citecolor=black,linkcolor=black,urlcolor=black}

% Metadaten der PDF-Datei
\hypersetup{
  pdftitle={Titel der Arbeit},
  pdfsubject={Seminararbeit},
  pdfauthor={Name des Autors / der Autorin},
  pdfcreator={},
  pdfproducer={}
}

% Zum schnellen Kompilieren während des Schreibens kann mit \includeonly{filename1,filename2,...} die Kompilierung auf bestimmte Kapitel beschränkt werden
% Damit die Referenzen aus den anderen Kapiteln korrekt sind, muss einmal mit allen Kapiteln kompiliert werden
% \includeonly{chapters/02_manual}  % nur Kapitel 2 wird kompiliert

\begin{document}

\include{chapters/frontmatter}

% Stelle die Sprache ein: "ngerman" für Deutsch, "english" für Englisch
\selectlanguage{ngerman}

%%%%%%%%%%%%%%%%%%%%%%%%%%%%%%%%%%%%%%%%%%%%%%%%%%%%%%%%%%%%%%%%%%%%%%%%%%%%%%%%%%%%%%%%%%%%%%%%%%%%%%%%%%%%%%
% INHALTSVERZEICHNIS [OBLIGATORISCH]
%%%%%%%%%%%%%%%%%%%%%%%%%%%%%%%%%%%%%%%%%%%%%%%%%%%%%%%%%%%%%%%%%%%%%%%%%%%%%%%%%%%%%%%%%%%%%%%%%%%%%%%%%%%%%%

\pagenumbering{roman}

\tableofcontents

\cleardoublepage
\pagenumbering{arabic}

%%%%%%%%%%%%%%%%%%%%%%%%%%%%%%%%%%%%%%%%%%%%%%%%%%%%%%%%%%%%%%%%%%%%%%%%%%%%%%%%%%%%%%%%%%%%%%%%%%%%%%%%%%%%%%
% KAPITEL DER ARBEIT
%%%%%%%%%%%%%%%%%%%%%%%%%%%%%%%%%%%%%%%%%%%%%%%%%%%%%%%%%%%%%%%%%%%%%%%%%%%%%%%%%%%%%%%%%%%%%%%%%%%%%%%%%%%%%%

% Verwende \include für modulare Struktur der Arbeit
% Es bietet sich an, jedes Kapitel in einem eigenen File zu schreiben und mit \include{filename} einzubinden
% \include fügt immer einen Seitenumbruch ein
% Mit \includeonly{filename1,filename2,...} können nur bestimmte files kompiliert werden (siehe oben)
% Eine fortgeschrittenere Alternative zu \include ist das Paket "subfiles"
\chapter{Einleitung}
\label{chapter:introduction}

\noindent
Die Kurzfassungen und die Einleitung sind unter Umständen die wichtigsten Teile der Arbeit, weil sie darüber entscheiden, ob jemand die Arbeit liest. Üblicherweise werden diese Abschnitte erst geschrieben, wenn der eigentliche Inhalt der Arbeit als Ganzes steht.

\begin{itemize}

    \item Die Einleitung soll an das Thema der Arbeit heranführen und die Hauptergebnisse und Hauptmethoden der Arbeit enthalten. Sie sollen Ihre Arbeit in den Kontext der mathematischen Literatur einordnen und Ihre wichtigsten Quellen angeben.

    \item Die Einleitung enthält alle Informationen, die bereits in der Kurzfassung stehen, gibt aber viel mehr Details und darf / sollte insbesondere auch Formeln enthalten.

    \item Worum geht es?
    \item Einordnung der Arbeit ins Forschungsfeld?
    \item Was ist die Fragestellung?
    \item Warum ist das wichtig?
    \item Was gibt es für Resultate in der vorhandenen Literatur?

    \item In der Einleitung müssen Sie den eigenen Anteil der Arbeit kenntlich machen:
          \begin{itemize}
              \item Wos woar mei Leistung?
              \item Was sind die Beiträge der vorliegenden Arbeit?
              \item Was ist besser als in bisherigen Arbeiten?
          \end{itemize}

    \item Was sind die eigenen Resultate?
          \begin{itemize}
              \item Ggf. Meta-Theoreme formulieren!
              \item Für genauere Formulierung des Theorems nach hinten verweisen!
          \end{itemize}

    \item Groben Aufbau der Arbeit skizzieren!
    \item Verweise auf Notationen / Resultate!
\end{itemize}
\chapter{Los geht's}
\label{chapter:content}

%%%%%%%%%%%%%%%%%%%%%%%%%%%%%%%%%%%%%%%%%%%%%%%%%%%%%%%%%%%%%%%%%%%%%%%%%%%%%%%%%%%%%%%%%%%%%%%%%%%%%%%%%%%%%%
\section{Hier beginnt die Arbeit}
%%%%%%%%%%%%%%%%%%%%%%%%%%%%%%%%%%%%%%%%%%%%%%%%%%%%%%%%%%%%%%%%%%%%%%%%%%%%%%%%%%%%%%%%%%%%%%%%%%%%%%%%%%%%%%

\begin{itemize}
  \item Üblicherweise mit den benötigten Notationen und aus der Literatur bekannten Resultaten etc.
  \item Grundlegende Definitionen und Resultate der Mathematik-Grundvorlesungen (z.B.\ Eigenwerte, Satz von Taylor, Banach-Räume etc.) dürfen Sie voraussetzen. Resultate der höheren Vorlesungen (z.B.\ Lemma von Lax--Milgram, Definition der Sobolev-Räume) können Sie (mit Referenz und \emph{ohne Beweis}) wiederholen.
  \item Wesentlich ist, dass Sie die benötigte Notation \emph{eindeutig} einführen. Verwenden Sie z.B.\ \emph{entweder} $W^{1,2}(\Omega)$ \emph{oder} $H^1(\Omega)$ für den Sobolev-Raum.
\end{itemize}

%%%%%%%%%%%%%%%%%%%%%%%%%%%%%%%%%%%%%%%%%%%%%%%%%%%%%%%%%%%%%%%%%%%%%%%%%%%%%%%%%%%%%%%%%%%%%%%%%%%%%%%%%%%%%%
\section{Ein paar allgemeine Hinweise}
%%%%%%%%%%%%%%%%%%%%%%%%%%%%%%%%%%%%%%%%%%%%%%%%%%%%%%%%%%%%%%%%%%%%%%%%%%%%%%%%%%%%%%%%%%%%%%%%%%%%%%%%%%%%%%

\begin{itemize}

  \item Regelmäßige Überprüfung der Rechtschreibung und Interpunktion!
  \item Schreiben Sie stets in vollständigen Sätzen und vermeiden Sie Abkürzungen.  Verwenden Sie eine verständliche Sprache mit klaren, nicht zu langen Sätzen. Es ist üblich, im Deutschen die Wir-Form zu verwenden (z.B.\ \emph{Im Folgenden beweisen wir...}).

  \item Achten Sie bei mathematischen Aussagen auf die sprachliche Reihenfolge etwaiger Quantoren (z.B.\ \emph{Auf jeden Topf passt ein Deckel!} vs.\ \emph{Es gibt einen Deckel, der auf jeden Topf passt!}).

  \item Achten Sie auf eine einheitliche Verwendung von Eigennamen: \emph{Dirichlet-Rand} oder \emph{Dirichletrand}!

  \item Saubere Trennung von mathematischer Aussage und Interpretation einer Aussage. Am besten vor jedem Satz einen kurzen Text über die Bedeutung dieses Ergebnisses.
  \item Jede "`Section"' beginnt mit einer kurzen Einleitung, was das Ziel und was die Hauptergebnisse dieser "`Section"' sind.

  \item Text sollte nie mit einer Formel beginnen.
        \begin{itemize}
          \item Besser: \emph{Es bezeichnet $\mathcal{S}^1(\mathcal{T}) := \big\{ ... \big\}$ den Raum ...}
          \item Anstelle von: \emph{$\mathcal{S}^1(\mathcal{T}) := \big\{ ... \big\}$ ist der Raum ...}
        \end{itemize}

  \item In Texten sollte nie Formel auf Formel folgen. Bitte ggf.\ den Text umformulieren!
        \begin{itemize}
          \item Besser: \emph{Für eine Triangulierung $\mathcal{T}$ definiere $\mathcal{S}^1(\mathcal{T}) := ...$}
          \item Anstelle von: \emph{Definiere für eine Triangulierung $\mathcal{T}$ $\mathcal{S}^1(\mathcal{T}) := ...$}
        \end{itemize}

  \item Lemmata, Sätze, Proposition etc.\ sollten alle gemeinsam gezählt/nummeriert werden (d.h.\ es gibt in dem Dokument \emph{nicht} Lemma~1 und Satz~1). Dadurch lassen sich Resultate im Dokument für den Leser leichter finden. Ggf.\ können Sie das Kapitel in den Zähler einbinden (z.B.\ Satz 3.5).
        \begin{itemize}
          \item Am einfachsten ist dazu die Verwendung von \verb$\newtheorem$. 
          Im Klassenfile sind bereits die wichtigsten Umgebungen definiert. 
          \item Verwendung: \verb$\begin{satz} ... \end{satz}$.
        \end{itemize}

  \item Nur wichtige Formeln kriegen eine Nummer. Das sind üblicherweise Formeln in Sät\-zen, Lemmata etc.\ sowie Formeln, auf die Sie im Beweis referenzieren.

  \item Vermeiden Sie die Verwendung von Quantoren und Folgt-Pfeilen im Fließtext, d.h.\ mischen Sie nicht logische Aussagen und Text (d.h.\ keine Verwendung von Quantoren zur Abkürzung von Text)! Quantoren und logische Symbole werden üblicherweise nur in abgesetzten Formeln verwendet.
        \begin{itemize}
          \item z.B. \emph{Damit haben wir die folgende Aussage gezeigt:
                  \begin{equation*}
                    \forall \varepsilon > 0 \, \exists h_0 > 0 \, \forall \mathcal{T}_h\text{ Triangulierung}:
                    \quad \Big( \| h \|_{L^\infty(\Omega)} \le h_0
                    \, \Longrightarrow \,
                    \| u - u_h \|_{H^1(\Omega)} \le \varepsilon \Big).
                  \end{equation*}}
        \end{itemize}

  \item Binden Sie Formeln als Teile von Sätzen ein. Achten Sie insbesondere auf die Zeichensetzung nach Formeln!
        \begin{itemize}
          \item Bespiel: \emph{Insgesamt erhalten wir damit
                  \begin{equation}\label{eq:hoelder}
                    \int_\Omega fg\,dx
                    \le \bigg(\int_\Omega f^p \, dx\bigg)^{1/p} \bigg(\int_\Omega g^q \, dx\bigg)^{1/q}.
                  \end{equation}}
        \end{itemize}

  \item Falls Sie auf eine Formel in einem Beweis an einer Stelle außerhalb des Beweises referenzieren müssen, so sollten Sie diese Formel als eigenes Lemma formulieren.

  \item Falls Ergebnisse/Argumente mehrfach verwendet werden, sollten Sie diese als eigenes Lemma formulieren.

  \item Machen Sie Ihre wesentlichen Ergebnisse (vor allem Sätze, aber auch Lemmata) zitierbar:
        \begin{itemize}
          \item alle Voraussetzungen hinschrieben (oder auf Generalvoraussetzung verweisen),
          \item bei etwaigen Konstanten in Abschätzungen die Abhängigkeit möglichst genau formulieren.
        \end{itemize}

\end{itemize}

%%%%%%%%%%%%%%%%%%%%%%%%%%%%%%%%%%%%%%%%%%%%%%%%%%%%%%%%%%%%%%%%%%%%%%%%%%%%%%%%%%%%%%%%%%%%%%%%%%%%%%%%%%%%%%
\section{Beweise}
%%%%%%%%%%%%%%%%%%%%%%%%%%%%%%%%%%%%%%%%%%%%%%%%%%%%%%%%%%%%%%%%%%%%%%%%%%%%%%%%%%%%%%%%%%%%%%%%%%%%%%%%%%%%%%

\begin{itemize}

  \item Verwenden Sie die Beweisumgebung \verb$\begin{proof} ... \end{proof}$.
        \begin{itemize}
          \item Verwenden Sie \verb$\begin{proof}[Text] ... \end{proof}$, 
          wenn Sie den Beweis nicht mit "`Beweis"' beginnen möchten.
          \item Beispiel: Wenn Sie "`Beweis zu Lemma~XY"' schreiben möchten.
        \end{itemize}

  \item Um die Lesbarkeit der Arbeit zu erhöhen, sollten Beweise deduktiv formuliert werden. Dadurch kann der Leser der Argumentation leichter folgen.
        \begin{itemize}
          \item Besser: \emph{Weil ..., gilt ...}
          \item Anstelle von: \emph{Es gilt ..., weil ...}
        \end{itemize}

  \item Schreiben Sie kurze Sätze mit maximal ein bis zwei Nebensätzen anstelle von langen Satzgefügen.
        \begin{itemize}
          \item Besser: \emph{Es gilt ... Daraus folgt ... Mithilfe von ... ergibt sich ...}
          \item Anstelle von: \emph{Es gilt ..., woraus wir zunächst ... und dann schließlich mithilfe von ... auch ... erhalten.}
        \end{itemize}

  \item Seien Sie eindeutig, aber nicht ausschweifend in Ihren Texten! Anders als bei einem Deutschaufsatz ist sprachliche Vielfalt nicht das Ziel und Wortwiederholung im Grunde OK! Dies gilt insbesondere dann, wenn ein formales Argument wiederholt wird (d.h.\ gleiche Argumente verwenden dieselben Worte).

  \item Arbeiten Sie die Beweise vollständig aus! Anders als in einem Paper gibt es in der Arbeit keine Begrenzung der Seitenzahl.

  \item Erklären Sie Rechnungen! Eine Rechnung, die über mehrere Zeilen geht, ist in der Regel nicht von alleine verständlich. Für jedes Argument gilt: Sie müssen es entweder zitieren oder beweisen. Standardargumente (z.B.\ Dreiecksungleichung, H\"older-Ungleichung) müssen lediglich genannt, aber nicht zitiert werden.

  \item Gliedern Sie längere Beweise \emph{sichtbar} in mehrere Schritte und sagen Sie explizit, was in jedem Schritt gezeigt wird.
        \begin{itemize}
          \item \emph{Schritt~1.} Zunächst zeigen wir...
          \item Verwenden Sie bitte keine \verb$itemize$-Umgebung für die Beweisschritte!
        \end{itemize}

  \item Erklären Sie, warum Sie "`o.B.d.A"' schreiben! Warum darf man das ohne Einschrän\-kung annehmen?

  \item Vermeiden Sie die Einschüchterung des Lesers ("`elementare Rechnung zeigt"', "`trivialerweise gilt"' etc.).
        \begin{itemize}
          \item Elementare Rechnungen können Sie ggf.\ in einem Appendix sammeln, wenn diese den Lesefluss stören würden (d.h.\ in einem eigenen Abschnitt am Ende der Arbeit, der mit \verb$\appendix$ \emph{vor} den folgenden Gliederungsbefehlen \verb$\chapter{...}$ bzw.\ \verb$\section{...}$ eingeleitet wird).
        \end{itemize}

  \item Bei Abschätzungen können Sie z.B.\ mittels \verb$\stackrel{\eqref{eq:formel}}{\le}$ beziehungsweise dem vordefinierten Makro \verb$\eqreff{eq:formel}{\le}$ Hinweise geben, dass diese Abschätzung aus der Formel mit \verb$\label{eq:formel}$ folgt, z.B.
        \begin{equation*}
          \textup{LHS} \eqreff{eq:hoelder}{\le} \textup{RHS}.
        \end{equation*}
        Dies ist für den Leser eine wesentliche Erleichterung, um umfangreiche Abschätzungen nachzuvollziehen.

  \item Bei mehrzeiligen Formeln sollten Sie die Gleichheitszeichen untereinander ausrichten. Dies erreichen Sie mittels der vordefinierten \verb$\eqreff*{eq:formel}{\le}$-Umgebung, z.B.
        \begin{align*}
          \textup{LHS} \  & \eqreff*{eq:hoelder}{\le} \ \textup{RHS}_1  \\
                          & \eqreff*{eq:hoelder}{\le} \ \textup{RHS}_2.
        \end{align*}
        Das ist insbesondere bei umfangreichen Abschätzungen sehr hilfreich. Sie müssen dann aber
        manuell Abstände vor und nach dem Gleichheitszeichen einfügen, z.B.\ \verb$LHS \ \eqreff*{eq:hoelder}{\le} \ RHS_1$.

  

  \item Ein Beweis sollte nie mit einer Formel enden, sondern mit Text, z.B.\ \emph{womit die Abschätzung {\normalfont(1.3)} gezeigt ist.}
\end{itemize}

%%%%%%%%%%%%%%%%%%%%%%%%%%%%%%%%%%%%%%%%%%%%%%%%%%%%%%%%%%%%%%%%%%%%%%%%%%%%%%%%%%%%%%%%%%%%%%%%%%%%%%%%%%%%%%
\section{Bilder}
%%%%%%%%%%%%%%%%%%%%%%%%%%%%%%%%%%%%%%%%%%%%%%%%%%%%%%%%%%%%%%%%%%%%%%%%%%%%%%%%%%%%%%%%%%%%%%%%%%%%%%%%%%%%%%

\begin{itemize}
  \item In den Bild-Unterschriften genau schreiben, was gezeigt wird, insbesondere zu welchem Abschnitt / Beispiel die Abbildung gehört.
        \begin{itemize}
          \item am besten auch kurze beschreibende Schlagwörter.
          \item Beispiel: \emph{Abbildung 3.5. Numerische Resultate zur singulären Lösung auf dem L-Shape aus  Abschnitt~3.1. Wir visualisieren den Fehler $\|u-u_h\|_{H^1(\Omega)}$ über der Anzahl $N$ der Elemente.}
        \end{itemize}

  \item Unterscheiden Sie bei der Beschreibung von Bildern, was Sie visualisieren, was Sie beobachten und wie Sie dies interpretieren (bzw.\ was dies belegt).

  \item Mathematische Grafiken können Sie leicht mithilfe von \verb$TikZ$ (siehe~\url{https://tikz.dev/tikz}) erstellen. Dafür sind die wichtigsten Einstellungen schon geladen. Eine nützliche Grafik könnte zum Beispiel die bekannte \texttt{SOLVE}--\texttt{ESTIMATE}--\texttt{MARK}--\texttt{REFINE} Schleife in Abbildung~\ref{fig:afem} sein.
        \begin{figure}
          \centering
          \begin{tikzpicture}
            \tikzset{myline/.style={draw=lightgray, line width=1.5pt, rounded corners}}
            \tikzset{afemnode/.style={myline, fill=lightgray!50, minimum width={width("ESTIMATE")+2pt}}}
            \node[afemnode] (solve) at (0,0) {\texttt{SOLVE}};
            \node[afemnode,right=of solve] (estimate) {\texttt{ESTIMATE}};
            \node[afemnode,right=of estimate] (mark) {\texttt{MARK}};
            \node[afemnode,right=of mark] (refine) {\texttt{REFINE}};
            %
            \tikzset{connection/.style={myline, gray, -stealth, , opacity=0.5}}
            \draw[connection] (solve) -- (estimate);
            \draw[connection] (estimate) -- (mark);
            \draw[connection] (mark) -- (refine);
            \draw[connection] (refine.north) -- +(0,15pt) -| ($(solve.north west)!0.33!(solve.north east)$);
            %
            \node at ($(estimate.east)!0.5!(mark.west)+(30pt,17pt)$) {\scriptsize solution accurate enough?};
          \end{tikzpicture}
          \caption{Die \texttt{SOLVE}--\texttt{ESTIMATE}--\texttt{MARK}--\texttt{REFINE} Schleife.\label{fig:afem}}
        \end{figure}

  \item In der Praxis ist es auch gut, wenn Sie die Grafiken in separate Dateien auslagern und diese dann in das Hauptdokument einbinden. Das erhöht die Übersichtlichkeit und erleichtert das Arbeiten an den einzelnen Kapiteln. Ein Beispiel dafür ist in Abbildung~\ref{fig:patch} gezeigt.
        \begin{figure}
          \centering
          \begin{tikzpicture}
            
	\node[regular polygon, regular polygon sides=6, fill = TUBlue!10!white, draw, minimum size=3cm](m) at (0,0) {};
	\node[above=4pt] (A) at (0,0) {$z$};	
	
	\draw [densely dotted, black] (m.corner 1) -- (0,0);
	\draw [densely dotted, black] (m.corner 2) -- (0,0);
	\draw [densely dotted, black] (m.corner 3) -- (0,0);
	\draw [densely dotted, black] (m.corner 4) -- (0,0);
	\draw [densely dotted, black] (m.corner 5) -- (0,0);
	\draw [densely dotted, black] (m.corner 6) -- (0,0);
	
	\fill [TUBlue] (0,0) circle (3pt);
	\draw [black] (0,0) circle (3pt);
	\fill [TUBlue] ($(m.corner 1)!0.5!(0,0)$) circle (3pt);
	\draw [black] ($(m.corner 1)!0.5!(0,0)$) circle (3pt);
	\fill [TUBlue] ($(m.corner 2)!0.5!(0,0)$) circle (3pt);
	\draw [black] ($(m.corner 2)!0.5!(0,0)$) circle (3pt);
	\fill [TUBlue] ($(m.corner 3)!0.5!(0,0)$) circle (3pt);
	\draw [black] ($(m.corner 3)!0.5!(0,0)$) circle (3pt);
	\fill [TUBlue] ($(m.corner 4)!0.5!(0,0)$) circle (3pt);
	\draw [black] ($(m.corner 4)!0.5!(0,0)$) circle (3pt);
	\fill [TUBlue] ($(m.corner 5)!0.5!(0,0)$) circle (3pt);
	\draw [black] ($(m.corner 5)!0.5!(0,0)$) circle (3pt);
	\fill [TUBlue] ($(m.corner 6)!0.5!(0,0)$) circle (3pt);
	\draw [black] ($(m.corner 6)!0.5!(0,0)$) circle (3pt);
	
	\node[right=1pt] (Omega) at (m.side 5) {$\omega_{L,z}$};
	
	\matrix [column sep=0.05, right = 65pt] (D) at (0,0)
	{
		\fill [TUBlue, centered] (0,0) circle (3pt); & \node[right]{degrees of freedom}; \\
		\node[centered]{$z$}; &  \node[right]{patch vertex in $\mathcal{V}_L$}; \\
		\node[centered]{$\cdots$}; & \node[right]{patch $\mathcal{T}_{L,z}$};  \\
		\node[centered]{---};  & \node[right]{patch subdomain $\omega_{L,z}$}; \\
	};
          \end{tikzpicture}
          \caption{Ein Patch $\omega_{L,z}$ mit zugehörigen Freiheitsgraden.}
          \label{fig:patch}
        \end{figure}

  \item Ein weiteres nützliches Tool für die Erzeugung von Grafiken ist pgfplots. Ein Manual dazu gibt es unter~\url{https://mirror.kumi.systems/ctan/graphics/pgf/contrib/pgfplots/doc/pgfplots.pdf}. Nützliche Einstellungen sind bereits im Klassenfile enthalten. Ein Beispiel für die Verwendung von pgfplots ist in Abbildung~\ref{fig:konvergenzplot} gezeigt.

        \begin{figure}
          \centering
          % Fügen Sie hier Ihre Daten ein, die Sie zum Beispiel aus Matlab oder Python 
% als csv Datei exportiert haben
\pgfplotstableread[col sep = comma]{plots/daten.csv}{\ErsterDatensatz}%

\begin{tikzpicture}[scale=1]
    \begin{loglogaxis}[
            xlabel = {Anzahl der Freiheitsgrade},%
            ylabel={Fehler},%
            line join = round,%
            line cap=round,%
            ymajorgrids      = true%
        ]

        \coordinate (legend) at (axis description cs:0.35,0.25);

        \addplot [pyBlue,thick, mark=*, mark options={solid, fill=pyBlue!50!white, scale = 1.1}, solid] table [x={nDofs}, y={estimator}]{\ErsterDatensatz}; \label{figure:ErsterDatensatz}

        % Referenzgerade
        \logLogSlope[reference]{0.1}{0.9}{0.7}{-1}{-5pt}{}
    \end{loglogaxis}

    % Erstelle Legende als Matrix an der Stelle (legend)
    \matrix [
        matrix of nodes,
        anchor=center,
        font=\scriptsize
    ] at (legend) {
        Fehlerschätzer & \\
        \ref{figure:ErsterDatensatz} & $p = 1$\\
    };
\end{tikzpicture}
          \caption{Numerische Resultate zur singulären Lösung auf dem
            L-Shape aus Abschnitt 3.1. Wir visualisieren den Fehlerschätzer $\eta_\ell$ über der
            Anzahl \texttt{ndof} der Freiheitsgrade.
          }
          \label{fig:konvergenzplot}
        \end{figure}
\end{itemize}

%%%%%%%%%%%%%%%%%%%%%%%%%%%%%%%%%%%%%%%%%%%%%%%%%%%%%%%%%%%%%%%%%%%%%%%%%%%%%%%%%%%%%%%%%%%%%%%%%%%%%%%%%%%%%%
\section{Algorithmen}
%%%%%%%%%%%%%%%%%%%%%%%%%%%%%%%%%%%%%%%%%%%%%%%%%%%%%%%%%%%%%%%%%%%%%%%%%%%%%%%%%%%%%%%%%%%%%%%%%%%%%%%%%%%%%%

\begin{itemize}
  \item Es gibt eine Vielzahl an Packages, die Umgebungen für Algorithmen bereitstellen.
  Wir stellen hier nur ein simples Beispiel einer Algorithmus-Umgebung vor, die im Klassenfile mittels \verb$newtheorem$ definiert ist.
\end{itemize}

\begin{algorithm}[2D newest vertex bisection]\label{alg:2d-nvb}
  \textbf{Input:} Triangulation $\mathcal{T}_\ell$ and set of marked elements $\mathcal{M}_\ell \subseteq \mathcal{T}_\ell$.
  \begin{enumerate}[topsep=0pt,itemsep=-1ex,partopsep=1ex,parsep=1ex]
    \item For all $T \in \mathcal{M}_\ell$, mark its refinement edge.
    \item \textbf{Repeat recursively}\label{alg-step:nvb-closure}
      \begin{itemize}[label={},leftmargin=*,labelindent=1ex,topsep=-1ex,itemsep=-1ex,partopsep=1ex,parsep=1ex]
        \item If a triangle $T \in \mathcal{T}_\ell$ has a marked edge that is not the refinement edge, also mark the refinement edge of $T$.
      \end{itemize}
      \textbf{until}
      \begin{itemize}[label={},leftmargin=*,labelindent=1ex,topsep=-1ex,itemsep=-1ex,partopsep=1ex,parsep=1ex]
        \item No triangle $T \in \mathcal{T}_\ell$ with marked edges has an unmarked refinement edge.
      \end{itemize}
    \item For all $T \in \mathcal{T}_\ell$, refine according to the pattern from Figure ?.
  \end{enumerate}
  \textbf{Output:} Refined triangulation $\mathcal{T}_{\ell+1}$.
\end{algorithm}

%%%%%%%%%%%%%%%%%%%%%%%%%%%%%%%%%%%%%%%%%%%%%%%%%%%%%%%%%%%%%%%%%%%%%%%%%%%%%%%%%%%%%%%%%%%%%%%%%%%%%%%%%%%%%%
\section{Tabellen}
%%%%%%%%%%%%%%%%%%%%%%%%%%%%%%%%%%%%%%%%%%%%%%%%%%%%%%%%%%%%%%%%%%%%%%%%%%%%%%%%%%%%%%%%%%%%%%%%%%%%%%%%%%%%%%
\begin{itemize}
  \item Bei Tabellen sollten Sie weder vertikale Linien noch doppelte horizontale Linien verwenden.
  \item Einen guten Überblick erhalten Sie in der Paketdokumentation von \verb$booktabs$ unter~\url{https://www.namsu.de/Extra/pakete/Booktabs.html}.
  \item Tabellen sollten eine klare Struktur haben und die Einträge sollten übersichtlich sein. Ein Beispiel für eine Tabelle ist in Tabelle~\ref{table:convergence} gezeigt.
  
  \begin{table}
    \centering
    \begin{booktabs}{
        colspec={cccccc},
        row{1}={font=\bfseries},
        column{1}={font=\bfseries},
    }
    \toprule
          & \SetCell[c=2]{c} Checkerboard && \SetCell[c=2]{c} Stripe\\
        \cmidrule[lr]{2-3} \cmidrule[lr]{4-5}
          & $p=1$                         & $p=2$                   & $p=1$   & $p=2$   \\
    \midrule
    $k=1$ & -0.4961                       & -0.9877                 & -0.4956 & -1.0116 \\
    $k=2$ & -0.4960                       & -0.9946                 & -0.4969 & -0.9670 \\
    $k=3$ & -0.4960                       & -0.9826                 & -0.5095 & -0.9766 \\
    \bottomrule
\end{booktabs}
    \caption{Experimentelle Konvergenzraten des Fehlerschätzers aus Abschnitt~3.1.}
    \label{table:convergence}
  \end{table}
\end{itemize}

%%%%%%%%%%%%%%%%%%%%%%%%%%%%%%%%%%%%%%%%%%%%%%%%%%%%%%%%%%%%%%%%%%%%%%%%%%%%%%%%%%%%%%%%%%%%%%%%%%%%%%%%%%%%%%
\section{Zitieren + Literaturverzeichnis}
%%%%%%%%%%%%%%%%%%%%%%%%%%%%%%%%%%%%%%%%%%%%%%%%%%%%%%%%%%%%%%%%%%%%%%%%%%%%%%%%%%%%%%%%%%%%%%%%%%%%%%%%%%%%%%

\begin{itemize}
  \item Um Ergebnisse aus Arbeiten zu zitieren, sollten Sie \verb$\cite[Theorem~X]{workY}$ nutzen und auch das konkrete Resultat angeben.
        \begin{itemize}
          \item z.B.\ \cite[Theorem~4]{MR3457440}
        \end{itemize}

  \item Wenn Sie ein Buch zitieren, geben Sie bitte mittels \verb$\cite[Section~X]{buchY}$
        den Abschnitt an. Häufig ist es besser, Abschnitte zu zitieren als Seitennummern.

  \item Verwenden Sie \verb$BibLaTeX$ (siehe \verb$literature.bib$ in diesem Template)!
        \begin{itemize}
          \item Alternativ können Sie auch die \verb$thebibliography$-Umgebung verwenden. Achten Sie in diesem Fall darauf, dass Sie die vollständigen bibliographischen Daten angeben (Artikel: Autoren, Titel, Zeitschrift, Band, Jahr, Seiten; Buch: Autoren, Titel, Verlag, Verlagsort, Auflage, Jahr), alle Einträge \emph{einheitlich} formatieren und alphabetisch sortieren.
        \end{itemize}

  \item Die \verb$BibLaTeX$-Einträge können Sie aus \url{http://www.ams.org/mathscinet/} mittels Copy'n'Paste überneh\-men.
  \item Auch wenn Sie die \verb$BibLaTeX$-Einträge aus \url{http://www.ams.org/mathscinet/} über\-neh\-men, sollten Sie darauf achten, dass am Ende die Einträge \emph{einheitlich} sind:
        \begin{itemize}
          \item Bei Vornamen von Autoren entweder alle abkürzen oder alle ausschreiben.
          \item Bei Journal-Namen entweder bei allen den vollständigen Namen angeben oder
                bei allen die offizielle Abkürzung gemäß \url{http://www.ams.org/mathscinet/} verwenden.
        \end{itemize}

\end{itemize}

%%%%%%%%%%%%%%%%%%%%%%%%%%%%%%%%%%%%%%%%%%%%%%%%%%%%%%%%%%%%%%%%%%%%%%%%%%%%%%%%%%%%%%%%%%%%%%%%%%%%%%%%%%%%%%
\section{\LaTeX}
%%%%%%%%%%%%%%%%%%%%%%%%%%%%%%%%%%%%%%%%%%%%%%%%%%%%%%%%%%%%%%%%%%%%%%%%%%%%%%%%%%%%%%%%%%%%%%%%%%%%%%%%%%%%%%

\begin{itemize}

  \item Alle \verb$Overfull \hbox$ und \verb$Overfull \vbox$ eliminieren!
        \begin{itemize}
          \item \verb$Overfull \hbox$ eliminiert man durch geeignete \verb$Sil\-ben\-tren\-nung$ oder geeignete \verb$\linebreak$.
          \item \verb$Overfull \vbox$ eliminiert man durch geeignete \verb$\pagebreak$.
        \end{itemize}

  \item Absätze macht man in \LaTeX, indem man Leerzeilen verwendet. Üblicherweise wird dadurch auch das erste Wort des neuen Absatzes eingerückt, was die Lesbarkeit erhöht. Bitte also \emph{nicht} \verb$\\$ oder \verb$\newline$ für einen neuen Absatz verwenden!
  \item Grundsätzlich machen Leerzeilen vor/nach Umgebungen (z.B.\ \verb$\begin{theorem}$ etc.) den Source-Code lesbarer.
  \item Das Tilde-Symbol verhindert einen Zeilenumbruch und wird deshalb vor \verb$\ref$, \verb$\cite$ etc.\ verwendet, z.B.\ \verb$Laut Satz~\ref{satz:xxx} gilt$...

  \item In Subscripts oder Superscripts verwendet man kein \verb$\frac$, sondern schreibt den Bruch aus:
        \begin{itemize}
          \item Also besser $\displaystyle \bigg(\sum_{T \in \mathcal T} \eta_T^2\bigg)^{1/2}$ anstelle von
                $\displaystyle \bigg(\sum_{T \in \mathcal T} \eta_T^2\bigg)^{\frac{1}{2}}$.
          \item Also besser $m_h^{i+1/2}$ anstelle von $m_h^{i+\frac{1}{2}}$.
        \end{itemize}

  \item Verwenden Sie im Fließtext möglichst kein \verb$frac$!
        \begin{itemize}
          \item Besser: $(n+1)^{-1}$ oder $1/(n+1)$.
          \item Anstelle von: $\frac{1}{n+1}$.
        \end{itemize}

  \item Falls Sie ggf.\ Abkürzungen verwenden, sollten Sie in \LaTeX\ \verb$ggf.\$ schreiben, damit der Punkt nicht als Satzende (= größerer Abstand) interpretiert wird.

  \item Mittels Verwendung von \verb$\input{kapitelX.tex}$ können Sie Ihr Dokument auf mehrere Dateien aufteilen (z.B.\ kapitelweise).

  \item Damit in Formeln Konstanten und Klammern nicht aneinander picken, sollten Sie mittels \verb$\,$ händisch Abstände einfügen, z.B. \verb$C \, h^\alpha$. Dasselbe gilt in Integranden, z.B. \verb$\int_\Omega f^2 \, dx$. \LaTeX\ ist an diesen Stellen sehr "`knausrig"' mit Abständen.
\end{itemize}

%%%%%%%%%%%%%%%%%%%%%%%%%%%%%%%%%%%%%%%%%%%%%%%%%%%%%%%%%%%%%%%%%%%%%%%%%%%%%%%%%%%%%%%%%%%%%%%%%%%%%%%%%%%%%%
\section{Englisch vs.\ Deutsch}
%%%%%%%%%%%%%%%%%%%%%%%%%%%%%%%%%%%%%%%%%%%%%%%%%%%%%%%%%%%%%%%%%%%%%%%%%%%%%%%%%%%%%%%%%%%%%%%%%%%%%%%%%%%%%%

\begin{itemize}

  \item Sie dürfen Ihre Arbeit sowohl auf Englisch als auch auf Deutsch verfassen!

  \item Bitte schreiben Sie nur auf Englisch, wenn Sie wissen, was Sie tun! Der Korrekturaufwand kann für beide Seiten immens sein! Insbesondere gelten für die Zeichensetzung im Englischen andere Regeln als im Deutschen.

\end{itemize}

%%%%%%%%%%%%%%%%%%%%%%%%%%%%%%%%%%%%%%%%%%%%%%%%%%%%%%%%%%%%%%%%%%%%%%%%%%%%%%%%%%%%%%%%%%%%%%%%%%%%%%%%%%%%%%
\section{Iterationen}
%%%%%%%%%%%%%%%%%%%%%%%%%%%%%%%%%%%%%%%%%%%%%%%%%%%%%%%%%%%%%%%%%%%%%%%%%%%%%%%%%%%%%%%%%%%%%%%%%%%%%%%%%%%%%%

\begin{itemize}

  \item Üblicherweise erfolgt das Lesen und die Korrektur der Arbeit iterativ.
  \item Bitte verwenden Sie das Makro \verb$\revision{...}$, um Änderungen \revision{farbig hervorzuheben}.
  Dieses Makro kann bei der nächsten Iteration mithilfe von \url{https://aldadic.github.io/cleantex/} entfernt werden.

\end{itemize}

\include{chapters/03_sinn}

%%%%%%%%%%%%%%%%%%%%%%%%%%%%%%%%%%%%%%%%%%%%%%%%%%%%%%%%%%%%%%%%%%%%%%%%%%%%%%%%%%%%%%%%%%%%%%%%%%%%%%%%%%%%%%
% LITERATURVERZEICHNIS MIT BIBLATEX --> MATHSCINET VERWENDEN
%%%%%%%%%%%%%%%%%%%%%%%%%%%%%%%%%%%%%%%%%%%%%%%%%%%%%%%%%%%%%%%%%%%%%%%%%%%%%%%%%%%%%%%%%%%%%%%%%%%%%%%%%%%%%%

\printbibliography

\end{document}
