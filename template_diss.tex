\documentclass[a4paper,11pt,bibliography=totoc,listof=totoc,headinclude=true,cleardoublepage=empty,oneside]{NumPDEsThesis}
% Option "oneside" für einseitigen Druck. Weglassen, falls die Arbeit doppelseitig gedruckt wird

% Stelle die Sprache ein: "ngerman" für Deutsch, "english" für Englisch
\selectlanguage{ngerman}

% Lade Literaturdatenbank
\addbibresource{literature.bib}

% Lade weitere Pakete
\usepackage{NumPDEsMacros} % in NumPDEsMacros.sty sind einige nützliche Befehle definiert

% Subfile-Paket für modulare Struktur
\usepackage{subfiles}

% Bevor die Arbeit gedruckt wird, sollten die Farben ausgeschaltet werden, indem die folgende Zeile auskommentiert wird.
% \hypersetup{colorlinks=false,citecolor=black,linkcolor=black,urlcolor=black,pagebackref=false}

\begin{document}

%%%%%%%%%%%%%%%%%%%%%%%%%%%%%%%%%%%%%%%%%%%%%%%%%%%%%%%%%%%%%%%%%%%%%%%%%%%%%%%%%%%%%%%%%%%%%%%%%%%%%%%%%%%%%%
% TITELSEITE [OBLIGATORISCH]
%%%%%%%%%%%%%%%%%%%%%%%%%%%%%%%%%%%%%%%%%%%%%%%%%%%%%%%%%%%%%%%%%%%%%%%%%%%%%%%%%%%%%%%%%%%%%%%%%%%%%%%%%%%%%%

\pagenumbering{Alph}

\begin{titlepage}
  \vspace*{-2cm}
  \begin{center}
    \includegraphics[width=0.45\textwidth]{figures/TULogo.pdf}
    \vskip 1cm
    {\LARGE D~\Large I~S~S~E~R~T~A~T~I~O~N}
    \vskip 8mm
    {\huge\bfseries\color{change}Titel \\[1ex] ggf.\ mehrzeilig}
    \vskip 1cm
    \large 
    ausgef\"uhrt zum Zwecke der Erlangung des akademischen Grades\\[1ex]
    eines Doktors der Naturwissenschaften unter der Leitung von\\[1ex]
    \vskip 0.5cm
    {\Large\bfseries\color{change}Name des Betreuers}\\[1ex]
    E101 -- Institut f\"ur Analysis und Scientific Computing, TU Wien\\[1ex]
    \vskip 0.75cm
    eingereicht an der Technischen Universit\"at Wien\\[1ex]
    Fakult\"at f\"ur Mathematik und Geoinformation\\[1ex]
    \vskip 0.15cm
    von\\[1ex]
    \vskip 0.15 cm
    {\Large\bfseries\color{change}Name des Autors}\\[1ex]
    Matrikelnummer: {\color{change}12345678}\\[1ex]
    {\color{change}Straße und Hausnummer}\\[1ex]
    {\color{change}PLZ und Ort}
  \end{center}
  
  \vskip 1cm
  
  Diese Dissertation haben begutachtet:
  \begin{enumerate}

  \item \bfseries\large{\color{change} Prof.\ Dr.\ X Y}\\
  \normalfont\small{\color{change} Institut für XY, Universität Z}
  
  \item \bfseries\large{\color{change} Prof.\ Dr.\ X Y}\\
  \normalfont\small{\color{change} Institut für XY, Universität Z}

  \item \bfseries\large{\color{change} Prof.\ Dr.\ X Y}\\
  \normalfont\small{\color{change} Institut für XY, Universität Z}

  \end{enumerate}
  
  \vfill
  
  \small
  Wien, am {\color{change} Datum} %\today
  \vspace*{-15mm}
\end{titlepage}

\cleardoublepage

%%%%%%%%%%%%%%%%%%%%%%%%%%%%%%%%%%%%%%%%%%%%%%%%%%%%%%%%%%%%%%%%%%%%%%%%%%%%%%%%%%%%%%%%%%%%%%%%%%%%%%%%%%%%%%
% KURZFASSUNG AUF DEUTSCH [OBLIGATORISCH]
%%%%%%%%%%%%%%%%%%%%%%%%%%%%%%%%%%%%%%%%%%%%%%%%%%%%%%%%%%%%%%%%%%%%%%%%%%%%%%%%%%%%%%%%%%%%%%%%%%%%%%%%%%%%%%

\chapter*{Kurzfassung}
\thispagestyle{empty}

\begin{itemize}
\item auf Deutsch (auch wenn die Arbeit auf Englisch ist)
\item ca.\ 250 Wörter, maximal 5000 Zeichen
\begin{itemize}
\item Worum geht es?
\item Einordnung der Arbeit ins Forschungsfeld?
\item Was ist die Fragestellung?
\item Was sind die Ergebnisse?
\item Warum ist das wichtig?
\end{itemize}
\item Möglichst nur Text und keine Formeln verwenden!
\item Dieser Text wird auch in TISS und der WWW-Seite der TU-Bibliothek verwendet.
\end{itemize}

\cleardoublepage

%%%%%%%%%%%%%%%%%%%%%%%%%%%%%%%%%%%%%%%%%%%%%%%%%%%%%%%%%%%%%%%%%%%%%%%%%%%%%%%%%%%%%%%%%%%%%%%%%%%%%%%%%%%%%%
% ABSTRACT = KURZFASSUNG AUF ENGLISCH [OBLIGATORISCH]
%%%%%%%%%%%%%%%%%%%%%%%%%%%%%%%%%%%%%%%%%%%%%%%%%%%%%%%%%%%%%%%%%%%%%%%%%%%%%%%%%%%%%%%%%%%%%%%%%%%%%%%%%%%%%%

\chapter*{Abstract}
\thispagestyle{empty}

{\selectlanguage{english}

\begin{itemize}
\item auf Englisch (auch wenn die Arbeit auf Deutsch ist)
\item sollte die Übersetzung der Kurzfassung sein!
\item ca.\ 250 Wörter, maximal 5000 Zeichen
\begin{itemize}
\item Worum geht es?
\item Einordnung der Arbeit ins Forschungsfeld?
\item Was ist die Fragestellung?
\item Was sind die Ergebnisse?
\item Warum ist das wichtig?
\end{itemize}
\item Möglichst nur Text und keine Formeln verwenden!
\item Dieser Text wird auch in TISS und der WWW-Seite der TU-Bibliothek verwendet.
\end{itemize}

}

\cleardoublepage

%%%%%%%%%%%%%%%%%%%%%%%%%%%%%%%%%%%%%%%%%%%%%%%%%%%%%%%%%%%%%%%%%%%%%%%%%%%%%%%%%%%%%%%%%%%%%%%%%%%%%%%%%%%%%%
% DANKSAGUNG / ACKNOWLEDGEMENT [OPTIONAL]
%%%%%%%%%%%%%%%%%%%%%%%%%%%%%%%%%%%%%%%%%%%%%%%%%%%%%%%%%%%%%%%%%%%%%%%%%%%%%%%%%%%%%%%%%%%%%%%%%%%%%%%%%%%%%%

\chapter*{Danksagung} %\chapter*{Acknowledgement}
\thispagestyle{empty}

\begin{itemize}
\item auf Deutsch oder Englisch
\item Die Danksagung (engl. {\em Acknowledgement}) ist optional und kann auch entfallen. Denken Sie ggf.\ an Ihre eigenen Eltern!

\item Falls die Arbeit durch eine Forschungsprojekt finanziert wurde, so ist jedenfalls der Fördergeber (z.B.\ FWF oder WWTF) mit Projektnummer und Projektname zu nennen.
\begin{itemize}
\item siehe z.B.\ Dissertation von Michele Ruggeri:
\item[] \href{https://publik.tuwien.ac.at/files/publik_252806.pdf}{\ttfamily https://publik.tuwien.ac.at/files/publik\_252806.pdf}
\end{itemize}

\end{itemize}

\vfill

Bitte die Endversion als hartgebundenes Exemplar für den Betreuer. Auf dem Buchrücken bitte die folgende Prägung:
\begin{itemize}
\item Vorname Nachname (links)
\item Jahr der Arbeit (mittig)
\item Dissertation (rechts)
\end{itemize}
Dankeschön!

\cleardoublepage

%%%%%%%%%%%%%%%%%%%%%%%%%%%%%%%%%%%%%%%%%%%%%%%%%%%%%%%%%%%%%%%%%%%%%%%%%%%%%%%%%%%%%%%%%%%%%%%%%%%%%%%%%%%%%%
% EIDESSTATTLICHE ERKLAERUNG [OBLIGATORISCH]
%%%%%%%%%%%%%%%%%%%%%%%%%%%%%%%%%%%%%%%%%%%%%%%%%%%%%%%%%%%%%%%%%%%%%%%%%%%%%%%%%%%%%%%%%%%%%%%%%%%%%%%%%%%%%%

\chapter*{Eidesstattliche Erkl\"arung}
\thispagestyle{empty}

\vspace*{2cm}

Ich erkl\"are an Eides statt, dass ich die vorliegende Dissertation selbstst\"andig und ohne fremde Hilfe verfasst, andere als die angegebenen Quellen und Hilfsmittel nicht benutzt bzw. die w\"ortlich oder sinngem\"a{\ss} entnommenen Stellen als solche kenntlich gemacht habe.

\vspace*{3cm}

\noindent
Wien, am {\color{change}Datum} %\today
%
\hfill 
%
\begin{minipage}[t]{5cm}
\centering
\underline{\hspace*{5cm}}\\
\small\color{change}Name des Autors
\end{minipage}

\cleardoublepage

%%%%%%%%%%%%%%%%%%%%%%%%%%%%%%%%%%%%%%%%%%%%%%%%%%%%%%%%%%%%%%%%%%%%%%%%%%%%%%%%%%%%%%%%%%%%%%%%%%%%%%%%%%%%%%
% INHALTSVERZEICHNIS [OBLIGATORISCH]
%%%%%%%%%%%%%%%%%%%%%%%%%%%%%%%%%%%%%%%%%%%%%%%%%%%%%%%%%%%%%%%%%%%%%%%%%%%%%%%%%%%%%%%%%%%%%%%%%%%%%%%%%%%%%%

\pagenumbering{roman}

\tableofcontents

\cleardoublepage
\pagenumbering{arabic} 

%%%%%%%%%%%%%%%%%%%%%%%%%%%%%%%%%%%%%%%%%%%%%%%%%%%%%%%%%%%%%%%%%%%%%%%%%%%%%%%%%%%%%%%%%%%%%%%%%%%%%%%%%%%%%%
% KAPITEL DER ARBEIT
%%%%%%%%%%%%%%%%%%%%%%%%%%%%%%%%%%%%%%%%%%%%%%%%%%%%%%%%%%%%%%%%%%%%%%%%%%%%%%%%%%%%%%%%%%%%%%%%%%%%%%%%%%%%%%

% Verwende subfiles für modulare Struktur der Arbeit
% Es bietet sich an, jedes Kapitel in einem eigenen File zu schreiben und mit \subfile{filename} einzubinden
% Die subfiles können auch einzeln kompiliert werden, indem sie als Hauptdatei gesetzt werden
\subfile{chapters/einleitung}
\subfile{chapters/manual}

%%%%%%%%%%%%%%%%%%%%%%%%%%%%%%%%%%%%%%%%%%%%%%%%%%%%%%%%%%%%%%%%%%%%%%%%%%%%%%%%%%%%%%%%%%%%%%%%%%%%%%%%%%%%%%
% LITERATURVERZEICHNIS MIT BIBLATEX --> MATHSCINET VERWENDEN
%%%%%%%%%%%%%%%%%%%%%%%%%%%%%%%%%%%%%%%%%%%%%%%%%%%%%%%%%%%%%%%%%%%%%%%%%%%%%%%%%%%%%%%%%%%%%%%%%%%%%%%%%%%%%%

\printbibliography

%%%%%%%%%%%%%%%%%%%%%%%%%%%%%%%%%%%%%%%%%%%%%%%%%%%%%%%%%%%%%%%%%%%%%%%%%%%%%%%%%%%%%%%%%%%%%%%%%%%%%%%%%%%%%%
% LEBENSLAUF / CV [OBLIGATORISCH]
%%%%%%%%%%%%%%%%%%%%%%%%%%%%%%%%%%%%%%%%%%%%%%%%%%%%%%%%%%%%%%%%%%%%%%%%%%%%%%%%%%%%%%%%%%%%%%%%%%%%%%%%%%%%%%

\chapter*{Curriculum Vitae}
\thispagestyle{empty}

\noindent
{\bfseries\Large Persönliche Daten}

\begin{tabular}{rp{.7\textwidth}}
\hspace*{.2\textwidth}&\\
Name & {\bfseries Dirk Praetorius} \\
Geburtsdatum & 10.01.1974 \\
Geburtsort & Neunkirchen / Saar \\
Nationalität & Deutsch \\
Email & \verb$dirk.praetorius@tuwien.ac.at$ \\
Homepage & \verb$https://www.tuwien.at/mg/asc/praetorius$ \\
\end{tabular}

\bigskip
\bigskip
\hrule
\bigskip
\bigskip

\noindent
{\bfseries\Large Ausbildung}

\begin{tabular}{rp{.7\textwidth}}
\hspace*{.2\textwidth}&\\
seit 09/2001 & Universitätassistent am Institut für Angewandte und Numerische Mathematik, TU Wien, Österreich \\
02/2000--08/2001 & DFG-Stipendiat am Lehrstuhl für Wissenschaftliches Rechnen, Christian-Albrechts-Universität zu Kiel, Deutschland \\ 
10/1996--06/2000 & Diplomstudium Mathematik, Christian-Albrechts-Universität zu Kiel, Deutschland \\
10/1993--01/2000 & Studium von Mathematik und Lateinischer Philologie für das gymnasiale Lehramt, Christian-Albrechts-Universität zu Kiel, Deutschland \\
06/1993 & Abitur, Meldorfer Gelehrtenschule, Meldorf, Deutschland \\
\end{tabular}

\bigskip
\bigskip
\hrule
\bigskip
\bigskip

\noindent
{\bfseries\Large Wissenschaftliche Publikationen}

\bigskip

\noindent
D.~Praetorius: \emph{Remarks and examples concerning distance ellipsoids}, Colloquium Mathematicum, 93 (2002), 41--53.

\vspace*{3cm}

\noindent
Wien, am {\color{change}Datum} %\today
%
\hfill 
%
\begin{minipage}[t]{5cm}
\centering
\underline{\hspace*{5cm}}\\
\small\color{change}Name des Autors
\end{minipage}

\end{document}
